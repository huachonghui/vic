% $Id$

It is well known that real-time multimedia applications prefer smooth and
predictable throughput to a TCP-like abrupt sending rate change. The current
release of VIC/RAT operates just on top of the RTP/RTCP architecture over UDP
protocol, not providing a fine-grained congestion control for its real-time
interactive streaming flows. This may work well only if the application's codecs
can maintain the play-out buffer such that the perceived user's delay is less
than 150 msec or so, which is not guaranteed at all times. If those applications
can generate smooth and predictable throughput, then we can precisely control
the play-out buffer so that the packet's end-to-end delay traverses the link can
be less than 150 msec: a tolerable maximum e2e delay for real-time interactive
multimedia streaming applications. 

To address this issue, Floyd et al. have proposed TCP-Friendly Rate-based
Congestion Control (TFRC)~\cite{FHPW00} which is also adopted in Datagram
Congestion Control Protocol (DCCP). Although TFRC has emerged as a de facto
standard to provide smooth and predictable throughput for such applications, we
have observed that a flow traverses a low statistically multiplexed network link
such as a DSL line using drop-tail queueing, TFRC traffic can starve TCP
traffic.  Rhee and Vojnovic also observed similar problems that the long-term
throughput imbalance was caused by the convexity of the TCP equation that they
have used and the different RTO measure. We set out to ease the throughput
differences by re-introducing a TCP-like Ack mechanism while retaining the TCP
throughput equation when computing the sending rate. The outcome, TCP-Friendly
Window-based Congestion Control (TFWC)~\cite{SH06}, showed that it is much
fairer than TFRC when competing with the same number of TCP flows, and showed
much simpler to implement in real-world applications such as VIC and RAT.  

For Google Summer of Code (GSoC) 2008, we would like to implement TFRC and TFWC
over VIC to show how the user's perceived delay can be improved by such
congestion control protocols providing smooth and predictable throughput without
starving sources using TCP. TFR(W)C will be implemented using UCL common library
by integrating the core parts of TFR(W)C algorithms in C/C++.

\vspace{-0.2in} 
\begin{itemize}

\item[-] TFRC paper~\cite{FHPW00} and TFWC paper~\cite{SH06}.

\item[-] GSoC Host Project:
\textsf{AVATS}\footnote{\textsf{http://www.cs.ucl.ac.uk/research/avats/}}

\item[-] GSoC Organization:
\textsf{OMII-UK}\footnote{\textsf{http://www.omii.ac.uk/}}

\end{itemize}
