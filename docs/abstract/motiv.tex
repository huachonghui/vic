% $Id$

\subsection{does congestion control ever needed?}

One might be able to think that a real-time interactive streaming application
does not ever need the congestion control mechanisms, but in fact we can do much
things using congestion control over those applications. The strong part of the
contribution is that we get the precise control about the end-to-end packet
delay in any kind of networks: congestion control mechanisms can provide the
relevant information to upper layer (e.g., multimedia codecs). Considering the
fact that the delay (and delay jitter) is one of the critical parameters that a
user might greatly care specially for the ``interactive" multimedia streaming
applications, the controlled delay using congestion control could bring a
significant advantage to such applications.

\subsection{then, why TFRC and why TFWC?}

As we pointed out already, our target applications prefer smooth and predictable
transmission rate to a TCP-like rate changing behavior. To achieve this goal,
Sally proposed TFRC~\cite{FHPW00}, a de facto standard, in 2000 and it is
gaining a great popularity: it is also adopted DCCP's CCID3, being standardized
by IETF.  However, we have recently observed that TFRC has some limitations such
that:

\vspace{-0.2in}
\begin{itemize}

\item It can produce uncontrolled throughput oscillation in a certain network
conditions~\cite{CH07}.

\item The TCP throughput equation has the convexity property which
resulting throughput imbalance~\cite{RX07}.

\item The difficulty of measuring RTT correctly can result in a long-term 
throughput imbalance~\cite{AS06}.

\end{itemize}

\vspace{-0.2in} 
All of the problems above essentially stem from the same basic cause: it is
difficult to build a rate-based protocol with a rate that is directly inverse
proportional to RTT while achieving stability in all conditions. All these
issues beg the question of why TFRC is rate-based at all?  We have proposed a
window-based version of TFRC, which uses a TCP-like Ack-clocking feature but
merges this with the use of the TCP throughput equation, ala TFRC, to directly
adjust the sending window size. The goal is to remedy the issues discussed above
which relate to the combination of rate-based control. Using a window makes the
RTT implicit in the Ack clock, and removing the need to be rate-based makes life
much simpler for application writers, as they no longer need to work around the
limitations of the OS's short duration timers. The detailed protocol description
and initial results can be found at~\cite{SH06}.

In next section, we describe our implementation proposal for GSoC 2008.
