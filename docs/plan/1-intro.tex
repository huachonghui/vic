% $Id$

\subsection{\label{ssec:project_overview}Overview}

VIC/RAT~\cite{MEDIATOOLS} is a media tool that can have interactive real-time
audio/video conference. These tools have been used in many relevant research
projects such as AccessGrid~\cite{ACCESSGRID}, AVATS~\cite{AVATS} and
UltraGrid~\cite{UltraGrid}. Currently, VIC/RAT is supported by AVATS
project~\cite{AVATS}. VIC supports many video codecs since its initial
development and implements RTP/RTCP protocol~\cite{RTP} over UDP to increase
interoperability. In this year's GSoC project, we would like to implement
congestion control protocols (TFRC and TFWC) over VIC. 

\subsection{\label{ssec:background}Background}

Recently, a TCP-Friendly Rate-based congestion control protocol, ala
TFRC~\cite{FHPW00}, has been proposed and it is now being standardized by IETF
under DCCP CCID3~\cite{CCID3}. The main advantages of using this protocol are 1)
it can generate smooth and predictable sending rate 2) it can help timely packet
delivery. However, TFRC might be too much aggressive against TCP sources in some
environment or it might be too much unresponsive to generate a useful sending
rate in some other network environment. To solve this issue, a window-based
version of TFRC has been proposed: TCP-Friendly Window-based Congestion Control
(TFWC)~\cite{SH06} and~\cite{CH07}.

\subsection{\label{ssec:motivation}Motivation}

The motivation of implementing congestion control mechanisms over such a
multimedia application (e.g.VIC or RAT) is that we can have a good
controllability at the sending/receiving buffer to make the overall end-to-end
packet delay be within a certain level of timing requirements. Considering the
fact that the delay (and delay jitter) is one of the critical parameters that a
user might greatly care, the controlled delay using congestion control mechanism
could bring a significant advantage to those applications.

\newpage
